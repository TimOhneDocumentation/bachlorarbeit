\chapter{Introduction}
\label{chap:introduction}
"Nature will reveal its secrets, but only if we ask the right questions."[Werner Heisenberg]
\newline
Progress in particle physics has always been driven by the desire to understand the fundamental building blocks of our universe.
\newline Our current best theory for the innnerworkings of our world,
the standart model of particle physics shows us, that the matter we see around us is mostly made up of down and up qurks and electrons.
Combinations of these quarks, held together by the strong nuclear force form the proton and neutron,the nuclei of the atoms that make up the matter of the everyday world.
Eventhough the Proton was discovered over a hundred years ago by Ernest Rutherford\autocite{discoveryProton}, it is still not fully understood.
\newline
Since the proton, unlike the electron is a composite particle, it follows that it has an internal structure.
The semantic meaning of size in the realm of particele physics is not as straight forward as in the macroscopic world.An answer to the question,
what is the size of the proton can be given 
by looking at the charge distribution of the proton, which defines the charge radius of the proton.
\newline
The proton radius measurment at AMBER at CERN aims to reselove a discrepency between the charge radius of the proton as measured by the Lamb shift in muonic and ordinary hydrogen and the electron-proton scattering experiments,
the so called proton radius puzzel.
\newline
To achieve this, the PRM experiment will measure the cross section of elastic scattering of muons on protons.
The scintillating fiber hodoscope is a key component of the PRM experiment, as it provides crucial time measuments of the incoming and scattered mouns, needed for the measurment of the proton radius\Autocite{ProposalAmber}.
\newline
This thesis will focus on the development of the FPGA driven frontend electronics of the scintillating fiber hodoscope 
for the proton radius measurment at AMBER at CERN,
especially on the development of the FPGA firmware required for the control of the CITIROC1A ASIC, a part of the readout and trigger electronic. 
\newline
%INNSERT: here a genera l overview of the thesis and its structure and reaseachr objectives will be given
 
