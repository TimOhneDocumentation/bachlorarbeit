\chapter{Introduction} \label{chap:introduction}
%Layout and structure of the thesis has to b explained
% the sectence especaill on the development of the FPGA firmwarecontroling the Citiroc1A has to be reworked
"What we observe is not nature itself, but nature exposed to our method of questioning."[Werner Heisenberg]\autocite{Heisenberg1958}
\newline
Progress in particle physics has always been driven by the desire to understand the fundamental building blocks of our universe.
\newline Our current best theory for the innner workings of our world,
the standard model of particle physics, tells us, that the matter we see around us is mostly made up of down and up quarks and electrons.
Combinations of these quarks, held together by the strong nuclear force form the proton and neutron, the nuclei of the atoms that make up the matter of the everyday world.\autocite{Workman:2836514}
Even though the proton was discovered over a century ago by Ernest Rutherford\autocite{discoveryProton}, it still holds several mysteries that continue to puzzle physicists.
One of these mysteries is the size of the proton.
\newline
The semantic meaning of size in the realm of particle physics is not as straight forward as in the macroscopic world. An answer to the question,
what is the size of the proton can be given by looking at the charge radius of the proton, first measured in 1956 by E. E. Chambers and R. Hofstadter.\autocite{Hofstadter1956}
\newline
The radius of the proton has been measured several times more since then, with different methods, such as electron-proton scattering experiments and the lamb shift in muonic and ordinary hydrogen.
The results of these measurements differ by five standard deviations, giving rise to the so called proton radius puzzle.\autocite{ProposalAmber}
\newline
The proton radius measurement (PRM) experiment at AMBER at CERN aims to help resolve this puzzle by measuring the proton radius with a new method,
 the elastic scattering of muons on protons.
\newline
To achieve this, the PRM experiment will measure the cross section of this scattering process and from this extract the form factors of the proton, which in turn allows for the calculation of the proton radius.
\newline
The scintillating fiber hodoscope (SFH) is a key component of the PRM experiment, as it provides precise time measurments of the incoming and scattered mouns,
needed for particle distinction in pile-ups in the Pixel detectors .\Autocite{InternalcommunicationIgor}%F
\newline
This thesis will focus on the development of the FPGA driven frontend electronics of the scintillating fiber hodoscope (SFH),
designed at the Technical University of Munich by the Physics Department E18 for the proton radius measurment at AMBER at CERN.
Especially on the development of the FPGA firmware required for the control and readout of the Citiroc1A ASIC,
a crucial part of the readout electronics.
\newline 
The framework of this thesis is structured in order to provide the neccessary background information for the development of the frontend electronics of the SFH.
\newline
The next chapter gives an general overview of the PRM experiment and the theoretical background of the proton radius measurment.
\newline 
In chapter \ref{cha:frontend} the frontend electronics of the SFH, with a focus on the functionality and behaviour of Citiroc1A ASIC are described.
\newline
Chapter \ref{cha:development} explains the developed FPGA firmware for the control and readout of the Citiroc1A ASIC, as well as give a short overview of the setup that was used for testing the developed firmware and characterizing the frontend electronics.
\newline
Chapter \ref{cha:results} presents the results of the testing of the developed firmware and the performance of the frontend electronics of the SFH.
\newline 
The final chapter of this thesis gives a short summary of the results and an outlook on the future development of the frontend electronics of the SFH,
neccessary for it to successfuly contribute to the resuliton of the proton radius puzzle.



 
